\chapter{随机事件与概率}
\section{随机事件间的关系及运算}
\subsection{事件间的运算规律}
\begin{enumerate}
    \item 交换律 $A \cup B = B \cup A,AB = BA$
    \item 结合律 $(A \cup B) \cup C = A \cup (B \cup C)$ $(AB)C = A(BC)$
    \item 分配律 $(A \cup B) \cap C = (A \cap C) \cup (B \cap C) = AC \cup BC$
          $(A \cap B) \cup C = (A \cup C) \cap (B \cup C) = (A \cup C)(B \cup C)$
    \item 德摩根律 $\overline{A \cup B}  = \overline A  \cap \overline B ,\overline {A \cap B}  = \overline A  \cup \overline B $
\end{enumerate}
\section{概率的定义}
\begin{enumerate}
    \item 排列:$P_n^m= A_n^m=n(n-1)(n-2) \cdots (n-m+1)=\frac{n!}{(n-m)!}$
    \item 组合:$C_n^m = \frac{P_n^m}{m!} = \frac{n(n - 1)(n - 2)....(n - m + 1)}{m!} = \frac{n!}{(n - m)m!}$
    \item 重复排列:$n^m$
    \item 重复组合:$C_{n + m - 1}^m$
\end{enumerate}
概率的公理化定义:设试验$E$的样本空间为$S$,对于$S$中的每一个事件$A$,都有一个实数$P(A)$与它对应,称为事件$A$的概率,如果集合函数$P(·)$满足下列条件:
\begin{enumerate}
    \item 非负性:对于每一个事件A,有$P(A) \ge 0$
    \item 规范性:对于必然事件S,有$P(S) = 1$
    \item 可列可加性:设$A_1,A_2,\cdots$是两两互不相容的事件,即$A_iA_j = \emptyset (i \ne j)$,则有
          $$P\left( {\bigcup\limits_{i = 1}^\infty  {A_i} } \right) = \sum\limits_{i = 1}^\infty  {P(A_i)}
          $$
\end{enumerate}

\section{概率的性质}
\begin{enumerate}
    \item $P(\emptyset ) = 0$
    \item (概率的有限可加性)若有限个${A_1},{A_2}, \cdots ,{A_n}$是两两互不相容的事件,则
          $$P\left( {\bigcup\limits_{i = 1}^n {A_i} } \right) = \sum\limits_{i = 1}^n {P(A_i)}
          $$
    \item 设$A,B$是两个事件且$A \subset B$,则$P(A) \leqslant P(B),\quad P(B - A) = P(B) - P(A) $
    \item (减法公式)设$A,B$是任意两个事件,有$P(A-B)=P(A)-P(AB)=P(A \overline B)$
    \item 对于任一事件$A$,有$P(A) \leq 1$
    \item (加法公式)对于任意两事件$A,B$有$P(A \cup B) = P(A) + P(B) - P(AB)$
    \item (半可加性)对于任意两事件$A,B$有$P(A \cup B) \leqslant P(A) + P(B)$
\end{enumerate}
推广  三个事件和的情况
\begin{equation}
    \begin{gathered}
        P({A_1} \cup {A_2} \cup {A_3})
        = P({A_1}) + P({A_2}) + P({A_3}) - P({A_1}{A_2}) - P({A_2}{A_3}) \hfill \\
        \quad  - P({A_1}{A_3}) + P({A_1}{A_2}{A_3}) \hfill \\
    \end{gathered}
\end{equation}

n个事件和的情况
\begin{equation}
    P({A_1} \cup {A_2} \cup  \cdots  \cup {A_n}) = \sum\limits_{i = 1}^n {P({A_i})}  - \sum\limits_{1 \leqslant i < j \leqslant n} {P({A_i}{A_j})}$$$$ + \sum\limits_{1 \leqslant i < j < k \leqslant n} {P({A_i}{A_j}{A_k})}  +  \cdots  + {( - 1)^{n - 1}}P({A_1}{A_2} \cdots {A_n})
\end{equation}

$P(AB)+P(A\overline{B})=P(A)$
\section{条件概率}
\begin{definition}
    设$A,B$是两个事件,且$P(A) > 0$,称$$P(B|A) = \frac{{P(AB)}}{{P(A)}}$$为在事件$A$发生的条件下事件$B$发生的条件概率。
\end{definition}
\begin{property} %TODO
    条件概率
    \begin{enumerate}
        \item 非负性 $P(B|A) \ge 0$
        \item 规范性 $P(S|A) = 1$
        \item $P(\left. {{A_1} \cup {A_2}} \right|B) = P(\left. {{A_1}} \right|B) + P(\left. {{A_2}} \right|B) - P(\left. {{A_1}{A_2}} \right|B)$
        \item $P( A |B) = 1 - P(\overline A|B)$
        \item 可列可加性 设$A_1,A_2,\cdots$是两两互不相容的事件 $P(\left. {\bigcup\limits_{i = 1}^\infty  {{A_i}} } \right.|B) = \sum\limits_{i = 1}^\infty  {P(\left. {{A_i}} \right|B)} $
    \end{enumerate}
\end{property}


\begin{theorem}[乘法定理]
    设$P(A) > 0$,则有$P(AB) = P(\left. B \right.|A)P(A)$

    设$A,B,C$为事件,且$P(AB)>0$,则有$P(ABC) = P(\left. C \right|AB)P(\left. B \right|A)P(A)$
\end{theorem}



推广:设${A_1},{A_2}, \cdots ,{A_n}$为$n$个事件,$n \ge 2$且$P({A_1}{A_2} \cdots {A_{n - 1}}) > 0$,则有
\begin{equation}
    \begin{gathered}
        P({A_1}{A_2} \cdots {A_n}) = P(\left. {{A_n}} \right|{A_1}{A_2} \cdots {A_{n - 1}}) \times  \hfill \\
        P(\left. {{A_{n - 1}}} \right|{A_1}{A_2} \cdots {A_{n - 2}}) \times  \cdots  \times P(\left. {{A_2}} \right|{A_1})P({A_1})
    \end{gathered}
\end{equation}

\subsection{全概率公式}
\begin{definition}[样本空间的划分]
    设$S$为试验$E$的样本空间,$B_1,B_2,\cdots,B_n$为$E$的一组事件,若$B_iB_j = \emptyset (i \ne j)$且$\bigcup\limits_{i = 1}^n {{B_i}}  = S$,则称$B_1,B_2,\cdots,B_n$为样本空间$S$的一个划分。
\end{definition}
\begin{theorem}[全概率公式]
    设试验$E$的样本空间为$S$,$A$为$E$的事件,$B_1,B_2,\cdots,B_n$为S的一个划分,且$P(B_i) > 0(i = 1,2,\cdots,n)$,则
    $$P(A) = \sum\limits_{i = 1}^n {P({B_i})P(\left. A \right|{B_i})}
    $$
\end{theorem}


注意
条件$B_1,B_2,\cdots,B_n$为样本空间的一个分割,可改成$B_1,B_2,\cdots,B_n$互不相容,且$A \subset \bigcup\limits_{i=1}^{n} B_i $,上述定理依然处成立。

\subsection{贝叶斯公式}
\begin{theorem}
    设试验$E$的样本空间为$S$,$A$为$E$的事件,$B_1,B_2,\cdots,B_n$为S的一个划分,且$P(A) > 0,P(B_i) > 0(i = 1,2,\cdots,n)$,则$$P({B_i}|\left. A \right.) = \frac{{P({B_i})P(\left. A \right|{B_i})}}{{\sum\limits_{j = 1}^n {P({B_j})P(\left. A \right|{B_j})} }}$$
\end{theorem}


\section{独立性}
\subsection{两个事件的独立性}
\begin{definition}
    设$A,B$是两个事件,如果$$P(AB) = P(A)P(B)$$则称事件$A$与$B$相互独立。
\end{definition}

等价条件 $P(B|A) = P(B) \quad P(B|\overline A)=P(B) \quad P(B|\overline A)=P(B|A)$

性质 若$A$与$B$相互独立,则有$A$与$\overline B$,$\overline A$与$B$,$\overline A$与$\overline B$也相互独立。

\subsection{多个事件相互独立}
\begin{definition}
    设$A,B,C$是三个事件,如果
    $$\left\{ \begin{gathered}
            P(AB) = P(A)P(B) \\
            P(BC) = P(B)P(C) \\
            P(AC) = P(A)P(C) \\
            P(ABC) = P(A)P(B)P(C) \\
        \end{gathered}  \right.$$
    则称事件$A,B,C$相互独立。
\end{definition}

推广:
设$A_1,A_2,\cdots,A_n$是$n$个事件,如果对于其中任意$k$个事件$A_{i_1},A_{i_2},\cdots,A_{i_k}(1 \le i_1 < i_2 < \cdots < i_k \le n)$,有$$P(A_{i_1}A_{i_2}\cdots A_{i_k}) = P(A_{i_1})P(A_{i_2})\cdots P(A_{i_k})$$则称事件$A_1,A_2,\cdots,A_n$相互独立。

两个结论:
1. 若事件$A_1,A_2,\cdots,A_n(n \ge 2)$相互独立,则其中任意$k(2 \le k \le n)$个事件也相互独立。
2. 若$n$个事件$A_1,A_2,\cdots,A_n(n \ge 2)$相互独立,则将$A_1,A_2,\cdots,A_n$中任意一个或几个换成它们的对立事件,所得到的$n$个事件也相互独立。