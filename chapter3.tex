\chapter{多维随机变量及其分布}
\section{多维随机变量及其联合分布}
\subsection{多维随机变量}
\begin{definition}
    如果$X_1(\boldsymbol{\omega}),X_2(\boldsymbol{\omega}),\cdots,X_n(\boldsymbol{\omega})$是定义在同一个样本空间$\Omega=\left\{\omega\right\}$上
    的$n$个随机变量,则称
    $$
        X(\omega)=(X_1(\omega),X_2(\omega),\cdots,X_n(\omega))
    $$
    为$n$维(或$n$元)随机变量或随机向量
\end{definition}
\subsection{联合分布函数}
\begin{definition}
    对任意的$n$个实数$x_1,x_2,\cdots,x_n$,$n$个事件$|X_1\leqslant x_1|,|X_2\leqslant x_2|,\cdots,| X_n\leqslant x_n|$同时发生的概率
    $$
        F(x_{1},x_{2},\cdots,x_{n})=P(X_{1}\leqslant x_{1},X_{2}\leqslant x_{2},\cdots,X_{n}\leqslant x_{n})
    $$
    称为$n$维随机变量$(X_1 ,X_2 ,\cdots,X_n)$的联合分布函数
\end{definition}


\begin{definition}[二维分布函数的定义]
    设$X,Y$是二维随机变量,对于任意实数$x,y$,二元函数:$F(x,y) = P\{ (X \le x) \cap (Y \le y)\}  = P( X \le x,Y \le y)$,称为二维随机变量$X,Y$的分布函数,或称为随机变量$X,Y$的联合分布函数,简称分布函数。
\end{definition}

\begin{property}分布函数的性质
    \begin{enumerate}
        \item 单调性 $F(x,y)$分别对$x$或$y$是单调不减的,即
              当$x_1 < x_2$时,$F(x_1,y) \le F(x_2,y)$
              当$y_1 < y_2$时,$F(x,y_1) \le F(x,y_2)$

        \item 有界性 对于任意的$x,y$,有$0 \le F(x,y) \le 1$,且
              $F(-\infty,y) = F(x,-\infty) = F(-\infty,-\infty) =F(-\infty,+\infty) = 0$
              $F(+\infty,+\infty) = 1$

        \item 右连续性 对每个变量都是右连续的,即
              $F(x+0,y) = F(x,y) $,$F(x,y+0) = F(x,y) $

        \item 非负性 对于任意的$(a,b),(c,d),a < b,c< d$,有$P(a < x \leq b,c < y \leq d)=F(b,d) - F(a,d) - F(b,c) + F(a,c) \ge 0$
    \end{enumerate}

\end{property}
\subsection{联合分布列}
\begin{definition}
    若二维随机变量$(X,Y)$所取的可能值是有限对或无限可列多对,则称$(X,Y)$为二维离散型随机变量.
\end{definition}

\paragraph{二维离散型随机变量的分布律}
设二维离散型随机变量$(X,Y)$所有可能取的值为$(x_i,y_j),i,j=1,2,\cdots$,记$P(X=x_i,Y=y_j)=p_{ij},i,j=1,2,\cdots$,称$p_{ij}$为二维离散型随机变量$(X,Y)$的分布律,或随机变量$X \text{和} Y$的联合分布律.
其中${p_{ij}} \ge 0,\;\;\sum\limits_{i = 1}^\infty  {\sum\limits_{j = 1}^\infty  {{p_{ij}}} }  = 1$

\subsection{联合密度函数}
\begin{definition}
    对于二维随机变量$(X,Y)$的分布函数$F(X,Y)$,如果存在非负的函数$f(x,y)$使对于任意$x,y$有$F(x,y) = \int_{-\infty}^y {\int_{-\infty}^x {f(u,v)dudv} } $,则称$(X,Y)$是连续型的二维随机变量,函数$f(x,y)$为变量$X \text{和} Y$的联合概率密度.
\end{definition}

\begin{property} 性质
    \begin{enumerate}
        \item 非负性 对于任意$x,y$有$f(x,y) \ge 0$
        \item 正则性 $F(\infty,\infty)=\int_{-\infty}^{+\infty} {\int_{-\infty}^{+\infty} {f(x,y)dxdy} } = 1$
        \item 设$G$是$xoy$平面上的一个区域,点$(X,Y)$落在$G$内的概率为$P( (X,Y) \in G)  = \iint\limits_G {f(x,y)dxdy}$
        \item 若$f(x,y)$在$(x,y)$连续,则有$\frac{{{\partial ^2}F(x,y)}}{{\partial x\partial y}} = f(x,y)$
    \end{enumerate}

\end{property}
\subsection{常用多维分布}
\paragraph{多项分布}
$$P({X_1} = {n_1},{X_2} = {n_2},...,{X_r} = {n_r}) = \frac{{n!}}{{{n_1}!{n_2}!...{n_r}!}}{p_1}^{{n_1}}{p_2}^{{n_2}}..{p_r}^{{n_r}}$$
\paragraph{多维超几何分布}
$$P({X_1} = {n_1},{X_2} = {n_2},...,{X_r} = {n_r}) = \frac{{{\text{C}}_{{N_1}}^{{n_1}}{\text{C}}_{{N_2}}^{{n_2}}...{\text{C}}_{{N_r}}^{{n_r}}}}{{{\text{C}}_N^n}}$$

\paragraph{多维均匀分布}
设 $D$是平面上的有界区域,其面积为 $S$,若二维随机变量$(X,Y)$具有概率密度$$f(x,y) = \left\{ \begin{gathered}
        \frac{1}{S},\quad (x,y) \in D,  \\
        0,\quad  else.  \\
    \end{gathered}  \right.$$
则称$(X,Y)$在$D$上服从均匀分布.

\paragraph{二维正态分布}
若二维随机变量$(X,Y)$具有概率密度
$$
    f(x,y) = \frac{1}{{2{\pi}{\sigma _1}{\sigma _2}\sqrt {1 - {\rho ^2}} }}{{\text{e}}^{\frac{{ - 1}}{{2(1 - {\rho ^2})}}\left[ {\frac{{{{(x - {\mu _1})}^2}}}{{\sigma _1^2}} - \frac{{2\rho (x - {\mu _1})(y - {\mu _2})}}{{{\sigma _1}{\sigma _2}}} + \frac{{{{(y - {\mu _2})}^2}}}{{\sigma _2^2}}} \right]}}$$

$$( - \infty  < x < \infty ,\; - \infty  < y < \infty )$$

其中${\mu _1},{\mu _2},{\sigma _1},{\sigma _2},\rho $均为常数,且${\sigma _1} > 0,{\sigma _2} > 0, - 1 < \rho  < 1$.则称$(X,Y)$服从参数为${\mu _1},{\mu _2},{\sigma _1},{\sigma _2},\rho $的二维正态分布,记为$(X,Y) \sim N({\mu _1},{\mu _2},{\sigma _1},{\sigma _2},\rho )$



\section{边际分布与随机变量的独立性}
\subsection{边际分布函数}
设 $F(x,\,y)$为随机变量$(X,\,Y)$的分布函数,则$F(x,y) = P( X \leqslant x,\,Y \leqslant y) $,令$y \to  + \infty $,称
$P( X \leqslant x)  = P( X \leqslant x,\,Y <  + \infty )  = F(x, + \infty )$为随机变量$(X,Y)$关于$X$的边际分布函数,记为$F_X(x)$.
同理令$x \to  + \infty $
$F_Y(y)=F(+ \infty,y )  = P( X < +\infty,\,Y \leqslant y )=P(Y \leqslant y)$
为随机变量$(X,Y)$关于$Y$的边际分布函数

\subsection{边际分布列}
设二维离散型随机变量$(X,Y)$的联合分布列为$P(X=x_i,Y=y_j)=p_{ij},i,j=1,2,\cdots$
记为
$${p_{i \bullet }} = \sum\limits_{j = 1}^\infty  {{p_{ij}}}  = P(X = {x_i}) ,\quad i = 1,2, \cdots$$
$${p_{ \bullet j}} = \sum\limits_{i = 1}^\infty  {{p_{ij}}}  = P(Y = {y_i}) ,\quad j = 1,2, \cdots$$
分别称$p_{i \bullet }$,$p_{ \bullet j}$为$(X,Y)$关于$X \text{和} Y$的边际分布列。

\subsection{边际密度函数}
对于连续型随机变量$(X,Y)$,设它的概率密度为$f(x,y)$,由于
$${F_X}(x) = F(x,\infty ) = \int_{ - \infty }^x {[\int_{ - \infty }^\infty  {f(x,y)dy} ]} dx$$
记${f_X}(x) = \int_{ - \infty }^\infty  {f(x,y)dy} $.(按照$X$型域求$y$的积分范围)
称其为随机变量$(X,Y)$关于$X$的边际概率密度

同理可得$Y$的边际概率密度${f_Y}(y) = \int_{ - \infty }^{ \infty } {f(x,y)dx} $

二维正态分布的两个边缘分布都是一维正态分布,并且都不依赖于参数$\rho$

\subsection{随机变量间的独立性}
设$F(x,y)$及$F_X(x)$,$F_Y(y)$分别为二维随机变量$(X,Y)$的分布函数及边际分布函数,若对于任意$x,y$有
$P(X \leqslant x,Y \leqslant y)=P(X \leqslant x)P(Y \leqslant y)$
即$F(x,y)=F_X(x)F_Y(y)$则称随机变量$X$与$Y$相互独立

说明:
\begin{enumerate}
    \item 若离散型随机变量$(X,Y)$的联合分布律为$P(X=x_i,Y=y_j)=p_{ij},i,j=1,2,\cdots$,$X$和$Y$相互独立$\Leftrightarrow$$P(X=x_i,Y=y_j)=P(X=x_i)P(Y=y_j),i,j=1,2,\cdots$

    \item 若连续型随机变量$(X,Y)$的联合概率密度为$f(x,y)$,边缘概率密度分别为$f_X(x),f_Y(y)$,则有$X$和$Y$相互独立$\Leftrightarrow$$f(x,y)=f_X(x)f_Y(y)$

    \item $X$和$Y$相互独立,则$f(X)$和$g(Y)$也相互独立。
\end{enumerate}

\section{多维随机变量函数的分布}
\subsection{多维离散随机变量函数的分布}
\paragraph{泊松分布的可加性}
设$X \sim P({\lambda _1}),Y \sim P({\lambda _2})$,且$X$与$Y$相互独立,则$Z=X + Y \sim P({\lambda _1} + {\lambda _2})$

\paragraph{离散场合下的卷积公式}

\paragraph{二项分布的可加性}
设$X \sim b(n,p),Y \sim b(m,p)$,且$X$与$Y$相互独立,则$Z=X + Y \sim b(n + m,p)$

\subsection{最大值与最小值的分布}
$M=max(X,Y),N=min(X,Y)$的分布

设$X$与$Y$是两个相互独立的随机变量,它们的分布函数分别为$F_X(x)$和$F_Y(y)$

则有
${F_{\max }}(z) = P\{ M \leqslant z\}= P\{ X \leqslant z,Y \leqslant z\} $
$= P\{ X \leqslant z\} P\{ Y \leqslant z\}= {F_X}(z){F_Y}(z)$


${F_{\min }}(z) = P\{ N \leqslant z\} = 1 - P\{ N > z\}$
$= 1 - P\{ X > z,Y > z\}$
$= 1 - P\{ X > z\}  \cdot P\{ Y > z\}$
$ = 1 - [1 - P\{ X \leqslant z\} ] \cdot [1 - P\{ Y \leqslant z\} ]$
$= 1 - [1 - {F_X}(z)][1 - {F_Y}(z)]$

推广 设${X_1},{X_2}, \cdots ,{X_n}$是n个相互独立的随机变量,它们的分布函数分别为$F_{{X_i}}(x),i = 1,2,\\ \cdots ,n$,则$M =max(X_1,X_2,\cdots,X_n)$及$N =min(X_1,X_2,\cdots,X_n)$的分布函数分别为
${F_{\max }}(z) = {F_{{X_1}}}(z) \cdot {F_{{X_2}}}(z) \cdots {F_{{X_n}}}(z)$
${F_{\min }}(z) = 1 - [1 - {F_{{X_1}}}(z)][1 - {F_{{X_2}}}(z)] \cdots [1 - {F_{{X_n}}}(z)]$
若${X_1},{X_2}, \cdots ,{X_n}$相互独立且具有相同的分布函数$F(x)$,则
${F_{\max }}(z) = {[F(z)]^n}$${F_{\min }}(z) = 1 - {[1 - F(z)]^n}$

    \subsection{连续场合的卷积公式}

    \begin{theorem}
        $Z = X+Y$的分布
        设$X$与$Y$的概率密度为$f(x,y)$,则$Z=X+Y$的分布函数为
        $${F_Z}(z) = P\{ Z \leqslant z\}= \iint\limits_{x + y \leqslant z} {f(x,y)}dxdy $$
        $x=u-y \Rightarrow \int_{ - \infty }^{ + \infty } {[\int_{ - \infty }^z {f(u - y,y)du} ]} dy$
        $ = \int_{ - \infty }^z {[\int_{ - \infty }^{ + \infty } {f(u - y,y)dy} } ]du$
        由此可得概率密度函数为
        $${f_Z}(z) = \int_{ - \infty }^{ + \infty } {f(z - y,y)dy} $$
        由于$X$与$Y$对称,${f_Z}(z) = \int_{ - \infty }^{ + \infty } {f(x,z - x)} dx$

        当$X$与$Y$独立时,$f_Z(z)$也可以表示为$${f_Z}(z) = \int_{ - \infty }^{ + \infty } {{f_X}(z - y){f_Y}(y)dy} $$
        或$${f_Z}(z) = \int_{ - \infty }^{ + \infty } {{f_X}(x){f_Y}(z - x)} dx$$

    \end{theorem}

    \paragraph{正态分布的可加性}
    设$X$,$Y$相互独立且$X \sim N({\mu _1},\sigma _1^2),Y \sim N({\mu _2},\sigma _2^2) $,则$Z=X+Y$仍然服从正态分布,且$Z \sim N({\mu _1} + {\mu _2},{\sigma _1}^2 + {\sigma _2}^2)$

    若相互独立的两个变量$X \sim N({\mu _1},\sigma _1^2),Y \sim N({\mu _2},\sigma _2^2)$,则$Z = aX \pm bY \sim N(a{\mu _1} \pm b{\mu _2},{a^2}\sigma _1^2 + {b^2}\sigma _2^2)$


    \paragraph{伽马分布的可加性}
    设$X \sim \Gamma (\alpha_1,{\lambda}),Y \sim \Gamma (\alpha_2,{\lambda})$,且$X$与$Y$相互独立,则$Z=X + Y \sim \Gamma (\alpha_1 + \alpha_2,\lambda)$

    \subsection{变量变换法}
    \paragraph{变量变换}
    设二维随机变量$(X,Y)$联合密度函数为$p(x,y)$.函数$\left\{ {\begin{array}{*{20}{c}}
            {u = u(x,y)} \\
            {v = v(x,y)}
        \end{array}} \right.$有连续偏导数,且存在唯一的反函数$\left\{ {\begin{array}{*{20}{c}}
            {x = x(u,v)} \\
            {y = y(u,v)}
        \end{array}} \right.$,其雅可比行列式
    $$J = \frac{{\partial (x,y)}}{{\partial (u,v)}}{\text{ = }}\left| {\begin{array}{*{20}{c}}
            {{x_u}} & {{y_u}} \\
            {{x_v}} & {{y_v}}
        \end{array}} \right| \ne 0$$
$\left\{ {\begin{array}{*{20}{c}}
            {U = U(x,y)} \\
            {V = V(x,y)}
        \end{array}} \right.$,则$U$与$V$的联合密度函数为$p(u,v) = p[x(u,v),y(u,v)]\left| J \right|$

    \paragraph{积的公式}
    设随机变量$X$和$Y$相互独立,密度函数分别为$p_X(x),p_Y(y)$,则$U = XY$的密度函数为$${p_U}(u){\text{ = }}\int_{ - \infty }^\infty  {{p_X}(\frac{u}{v}){p_Y}(v)\frac{1}{{\left| v \right|}}} dv$$

    \paragraph{商的公式}
    设随机变量$X$和$Y$相互独立,密度函数分别为$p_X(x),p_Y(y)$,则$U = \frac{X}{Y}$的密度函数为$${p_U}(u){\text{ = }}\int_{ - \infty }^\infty  {{p_X}(uv){p_Y}(v)\left| v \right|} dv$$

    \section{多维随机变量的特征数}
    \subsection{多维随机变量函数的数学期望}
    \begin{theorem}
        设二维随机变量$(X,Y)$的联合分布列为$P(X = {x_i},Y = {y_j}) = {p_{i,j}}$或联合密度函数为$p(x,y)$,则$Z = g(X,Y)$的数学期望为$$E(Z) = \left\{ \begin{gathered}
                \sum\limits_{i,j} {g({x_i},{y_j}){p_{i,j}}} \quad \text{离散场合} \\
                \int_{\text{R}}^{} {\int_{\text{R}}^{} {g(x,y)p(x,y)dxdy} }  \quad \text{连续场合} \\
            \end{gathered}  \right.$$
    \end{theorem}

    \subsection{数学期望与方差的运算性质}
    \begin{enumerate}
        \item $E(X + Y) = E(X) + E(Y)$
        \item 若$X$,$Y$独立,则$E(XY) = E(X)E(Y)$
        \item 若$X$,$Y$独立,则$Var(X \pm Y) = Var(X) + Var(Y)$
        \item 推广:若$X_1,X_2,\cdots,X_n$相互独立,则$Var(\sum\limits_{i = 1}^n {{X_i}} ) = \sum\limits_{i = 1}^n {Var({X_i})} $
    \end{enumerate}

    \subsection{协方差}
    \begin{definition}
        设$(X,Y)$是二维随机变量,若$E[(X - E(X))(Y - E(Y))]$存在,则称其为$X$与$Y$的协方差,记为$Cov(X,Y)$,即$$Cov(X,Y) = E[(X - E(X))(Y - E(Y))]$$特别有$Cov(X,X) = Var(X)$
    \end{definition}

    注意:

$Cov(X,Y)>0$,$X$与$Y$正相关

$Cov(X,Y)<0$,$X$与$Y$负相关

$Cov(X,Y)=0$,$X$与$Y$不相关


    注:这里指的是线性相关,即$Y=aX+b$,可能有其他的相关关系,比如$Y=X^2$,这种情况下$Cov(X,Y)=0$,但是$X$与$Y$不独立。
    \begin{property} 协方差性质
        \begin{enumerate}
            \item $Cov(X,Y) = E(XY) - E(X)E(Y)$
            \item $X$,$Y$相互独立,则$Cov(X,Y) = 0$,反之不成立。
            \item 任意$X,Y$,有$Var(X \pm Y)=Var(X)+Var(Y) \pm 2Cov(X,Y) $
            \item $Cov(X,Y) = Cov(Y,X)$
            \item $Cov(X,a)=0$
            \item $Cov(aX,bY)=abCov(X,Y)$
            \item $Cov(X+Y,Z)=Cov(X,Z)+Cov(Y,Z)$
        \end{enumerate}

    \end{property}

    \subsection{相关系数}
    \begin{definition}
        设$(X,Y)$是二维随机变量,若$Var(X) > 0,Var(Y) > 0$,则称$\frac{{Cov(X,Y)}}{{\sqrt {Var(X)} \sqrt {Var(Y)} }}$为随机变量$X$与$Y$的相关系数,记为$\rho_{xy}$或$Corr(X,Y)$
    \end{definition}

    \begin{lemma}
        对任意二维随机变量$(X,Y)$,若$X$和$Y$的方差都存在,且记$\sigma_x^2=Var(X),\sigma_y^2=Var(Y)$,则有$[Cov(X,Y)]^2 \leqslant \sigma_x^2 \sigma_y^2$
    \end{lemma}

    注意:

    当$|\rho_{xy}|$较大时,表面$X$与$Y$的线性关系较紧密

    当$|\rho_{xy}|$较小时,表面$X$与$Y$的线性关系较弱

\begin{property} 性质
    \begin{enumerate}
        \item $\left| {{\rho _{XY}}} \right| \leqslant 1$
        \item $\left| {{\rho _{XY}}} \right| = 1 \Leftrightarrow $存在常数$a,b$,使得$P(Y = aX + b) = 1$
              $\rho_{XY}=1$时,$a>0$,$\rho_{XY}=-1$时,$a<0$
    \end{enumerate}

\end{property}

\begin{theorem}
    $X$,$Y$相互独立$\Rightarrow$$X$,$Y$不相关,反之不一定成立。
        如果$(X,Y)$服从二维正态分布,则$X$,$Y$不相关一定相互独立。
\end{theorem}


\begin{theorem}
    1. 若$\rho_{xy}=0$,则称$X$与$Y$不相关
    2. 若$\rho_{xy}>0$,则称$X$与$Y$正相关
    3. 若$\rho_{xy}<0$,则称$X$与$Y$负相关
\end{theorem}


\begin{theorem}
    若$(X,Y) \sim N({\mu _1},{\mu _2},\sigma _1^2,\sigma _2^2,\rho )$,则$Cov(X,Y) = \rho {\sigma _1}{\sigma _2}$,${\rho _{XY}} = \rho $
$X$与$Y$独立$\Leftrightarrow \rho=0$
\end{theorem}
\subsection{随机向量的数学期望与协方差矩阵}

\section{条件分布与条件期望}
\subsection{条件分布}
离散型随机变量的条件分布
\begin{definition}
    设$(X,Y)$是二维随机离散型变量,对于固定的$j$,若$P(Y=y_j)>0$,则称
$P( \left. {X = {x_i}} \right|Y = {y_j})  = \frac{ P( X = {x_i},Y = {y_j}) }{P( Y = {y_j}) } = \frac{ p_{ij} }{ p_{ \bullet j} }$为在$Y=y_j$条件下随机变量$X$的条件分布列
\end{definition}

对于固定的$i$,若$P(X=x_i)>0$,则称
$P( \left. {Y = {y_j}} \right|X = {x_i})  = \frac{ P( X = {x_i},Y = {y_j}) }{P( X = {x_i}) } = \frac{ p_{ij} }{ p_{ i \bullet} }$为在$X=x_i$条件下随机变量$Y$的条件分布列
其中,$i,j = 1,2, \cdots $

\begin{definition}[离散型随机变量的条件分布函数]
    给定$Y = {y_j}$条件下$X$的条件分布函数为
$$F(x\left| {{y_j})} \right. = \sum\limits_{{x_i} \leqslant x} {P(X = {x_i}\left| {Y = {y_j}) = } \right.} \sum\limits_{{x_i} \leqslant x} {{p_{i\left| j \right.}}} $$
给定$X = {x_i}$条件下$Y$的条件分布函数为
$$F(y\left| {{x_i})} \right. = \sum\limits_{{y_j} \leqslant y} {P(Y = {y_j}\left| {X = {x_i}) = } \right.} \sum\limits_{{y_j} \leqslant y} {{p_{j\left| i \right.}}} $$
    
\end{definition}

连续型随机变量的条件分布
\begin{definition}
    设二维随机变量$(X,Y)$的概率密度为$f(x,y)$,$(X,Y)$关于$Y$的边缘概率密度为${f_Y}(y)$.若对于固定的$y$,有$f_Y(y)>0$,则称$\frac{{f(x,y)}}{{{f_Y}(y)}}$为在$Y=y$条件下$X$的条件概率密度,记为
$${f_{_{\left. X \right|Y}}}(\left. x \right|y) = \frac{{f(x,y)}}{{{f_Y}(y)}}$$
称$\int_{ - \infty }^x {{f_{\left. X \right|Y}}(\left. x \right|y)dx = \int_{ - \infty }^x {\frac{{f(x,y)}}{{{f_Y}(y)}}} } dx$为在$Y=y$条件下,$X$的条件分布函数,记为$P( \left. {X \leqslant x} \right|Y = y) $或$F_{X|Y}(x|y)$
同理定义在$X=x$条件下$Y$的条件分布函数为
$${F_{\left. Y \right|X}}(\left. y \right|x) = P(\left. {Y \leqslant y} \right|X = x)  = \int_{ - \infty }^y {\frac{{f(x,y)}}{{{f_X}(x)}}} dy$$
\end{definition}

连续场合的全概率公式和贝叶斯公式

$p(x,y)=p_X(x)p(y|x)$

$p(x,y)=p_Y(y)p(x|y)$

$p_Y(y)=\int_{- \infty}^{\infty} p_X(x)p(y|x)dx$

$p_X(x)=\int_{- \infty}^{\infty} p_Y(y)p(x|y)dy$

$p(x|y)=\frac{p_X(x)p(y|x)}{\int_{- \infty}^{\infty} p_X(x)p(y|x)dx}$

$p(y|x)=\frac{p_Y(y)p(x|y)}{\int_{- \infty}^{\infty} p_Y(y)p(x|y)dy}$


\subsection{条件数学期望}

\begin{definition}
    条件分布的数学期望(若存在),称为条件期望,定义如下:
$$E(X\left| {Y = y} \right.) = \left\{ \begin{gathered}
  \sum\limits_i {{x_i}P(X = {x_i}\left| {Y = y)\quad (X,Y)离散} \right.}  \hfill \\
  \int_{ - \infty }^\infty  {xf(x\left| y \right.)dx \quad (X,Y)连续}  \hfill \\ 
\end{gathered}  \right.$$
$$E(Y\left| {X = x} \right.) = \left\{ \begin{gathered}
  \sum\limits_j {{y_j}P(Y = {y_j}\left| {X = x) \quad (X,Y)离散} \right.}  \hfill \\
  \int_{ - \infty }^\infty  {yf(y\left| x \right.)dy \quad (X,Y)连续}  \hfill \\ 
\end{gathered}  \right.$$

\end{definition}

\begin{theorem}
    重期望公式 设$(X,Y)$是二维随机变量,且$E(X)$存在,则$ E(X) = E[E(X|Y)]$
\end{theorem}

注:

若$Y$是离散型随机变量,则$$E(X){\text{ = }}\sum\limits_j {E(X\left| {Y = {y_j})P(} \right.} Y = {y_j})$$

若$Y$是连续型随机变量,则
$$E(X){\text{ = }}\int_{\text{R}}^{} {E(X} \left| {Y = y){f_Y}(y)dy} \right.$$


(随机个随机变量和的数学期望) 设${X_1},{X_2}, \cdots $为一系列独立同分布的随机变量,随机变量$N$只取正整数,且$N$与$\{X_n\} $独立,则$$E(\sum\limits_{i = 1}^N {{X_i}} ) = {\kern 1pt} E({X_1})E(N)$$